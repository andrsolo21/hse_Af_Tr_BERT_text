%!TEX TS-program = xelatex
\documentclass[a4paper,14pt]{article}

\input{data/preambular.tex}
\begin{document} 	
	
	\addcontentsline{toc}{section}{Приложение А}

\begin{flushright}
	Приложение А
\end{flushright}


\begin{center}
	Пример визуализации функции позиционного кодирования
\end{center}


\begin{figure}[H]
	\centering
	\includegraphics[width=0.95\linewidth]{image/v9dvohtljexbjop_vrykyyqdzbk}
	\caption*{Рис. А1. Пример визуализации функции позиционного кодирования}
	\label{fig:v9dvohtljexbjopvrykyyqdzbk}
\end{figure}

\newpage

\addcontentsline{toc}{section}{Приложение Б}

\begin{flushright}
	Приложение Б
\end{flushright}

\begin{center}
	Ссылки на исходные коды
\end{center}

Ссылка на github репозиторий \href{https://github.com/andrsolo21/hse_Af_Tr_BERT}{https://github.com/andrsolo21/hse\_Af\_Tr\_BERT}.

Ссылка на ноутбук с примером работы модели BERT: \href{https://github.com/andrsolo21/hse_Af_Tr_BERT/blob/main/new_hse_BERT_0.ipynb}{new\_hse\_BERT\_0.ipynb}.

Ссылка на ноутбук с примером работы модели ELMo:  \href{https://github.com/andrsolo21/hse_Af_Tr_BERT/blob/main/new_hse_ELMO_0.ipynb}{new\_hse\_ELMO\_0.ipynb}.

Ссылка на ноутбук с агрегацией результатов: \href{https://github.com/andrsolo21/hse_Af_Tr_BERT/blob/main/Visualization.ipynb}{Visualization.ipynb}.

\newpage

\addcontentsline{toc}{section}{Приложение В}

\begin{flushright}
	Приложение В
\end{flushright}

\begin{center}
	Визуализация эмбеддингов
\end{center}


\begin{figure}[H]
	\centering
	\includegraphics[width=0.95\linewidth]{image/pril_2}
	\caption*{Рис. В1. Визуализация эмбеддингов из группы <<прилагательное -- сравнительная степень>> для модели BERT с обработкой N-грамм через вычисление среднего эмбеддинга; Слева модель bert-base-multilingual-cased, справа от DeepPavlov; Верхние модели обучены на жанре литературы, нижние на политике}
	\label{fig:pril22}
\end{figure}

\begin{figure}[H]
	\centering
	\includegraphics[width=0.95\linewidth]{image/pril_3}
	\caption*{Рис. В2. Визуализация эмбеддингов из группы <<прилагательное -- сравнительная степень>> для модели BERT с обработкой N-грамм через вычисление суммы эмбеддингов; Слева модель bert-base-multilingual-cased, справа от DeepPavlov; Верхние модели обучены на жанре литературы, нижние на политике}
	\label{fig:pril32}
\end{figure}

\begin{figure}[H]
	\centering
	\includegraphics[width=0.95\linewidth]{image/elmo_pil}
	\caption*{Рис. В3. Визуализация эмбеддингов из группы <<прилагательное -- сравнительная степень>> для модели ELMo; Слева модель обучена на жанре литературы, справа на политике}
	\label{fig:elmopil2}
\end{figure}

\begin{figure}[H]
	\centering
	\includegraphics[width=0.95\linewidth]{image/fam_2}
	\caption*{Рис. В4. Визуализация эмбеддингов из группы <<мужской пол -- женский пол>> для модели BERT с обработкой N-грамм через вычисление среднего эмбеддинга; Слева модель bert-base-multilingual-cased, справа от DeepPavlov; Верхние модели обучены на жанре литературы, нижние на политике}
	\label{fig:fam2}
\end{figure}

\begin{figure}[H]
	\centering
	\includegraphics[width=0.95\linewidth]{image/fam_3}
	\caption*{Рис. В5. Визуализация эмбеддингов из группы <<мужской пол -- женский пол>> для модели BERT с обработкой N-грамм через вычисление суммы эмбеддингов; Слева модель bert-base-multilingual-cased, справа от DeepPavlov; Верхние модели обучены на жанре литературы, нижние на политике}
	\label{fig:fam3}
\end{figure}

\newpage

\addcontentsline{toc}{section}{Приложение Г}

\begin{flushright}
	Приложение Г
\end{flushright}

\begin{center}
	 Результаты тестирования моделей
\end{center}

\begin{figure}[H]
	\centering
	\includegraphics[width=0.9\linewidth]{image/res_bert-base-multilingual-cased-liter }
	\caption*{Рис. Г1. Результаты тестирования для модели bert-base-multilingual-cased-liter }
	\label{fig:resbert-base-multilingual-cased-liter }
\end{figure}

\begin{figure}[H]
	\centering
	\includegraphics[width=0.9\linewidth]{image/res_bert-base-multilingual-cased-liter2 }
	\caption*{Рис. Г2. Результаты тестирования для модели bert-base-multilingual-cased-liter2 }
	\label{fig:resbert-base-multilingual-cased-liter2 }
\end{figure}

\begin{figure}[H]
	\centering
	\includegraphics[width=0.9\linewidth]{image/res_bert-base-multilingual-cased-liter3 }
	\caption*{Рис. Г3. Результаты тестирования для модели bert-base-multilingual-cased-liter3 }
	\label{fig:resbert-base-multilingual-cased-liter3 }
\end{figure}

\begin{figure}[H]
	\centering
	\includegraphics[width=0.9\linewidth]{image/res_bert-base-multilingual-cased-polit }
	\caption*{Рис. Г4. Результаты тестирования для модели bert-base-multilingual-cased-polit }
	\label{fig:resbert-base-multilingual-cased-polit }
\end{figure}

\begin{figure}[H]
	\centering
	\includegraphics[width=0.9\linewidth]{image/res_bert-base-multilingual-cased-polit2 }
	\caption*{Рис. Г5. Результаты тестирования для модели bert-base-multilingual-cased-polit2 }
	\label{fig:resbert-base-multilingual-cased-polit2 }
\end{figure}

\begin{figure}[H]
	\centering
	\includegraphics[width=0.9\linewidth]{image/res_bert-base-multilingual-cased-polit3 }
	\caption*{Рис. Г6. Результаты тестирования для модели bert-base-multilingual-cased-polit3 }
	\label{fig:resbert-base-multilingual-cased-polit3 }
\end{figure}

\begin{figure}[H]
	\centering
	\includegraphics[width=0.9\linewidth]{image/res_DeepPavlov-liter }
	\caption*{Рис. Г7. Результаты тестирования для модели DeepPavlov-liter }
	\label{fig:resDeepPavlov-liter }
\end{figure}

\begin{figure}[H]
	\centering
	\includegraphics[width=0.9\linewidth]{image/res_DeepPavlov-liter_2 }
	\caption*{Рис. Г8. Результаты тестирования для модели DeepPavlov-liter\_2 }
	\label{fig:resDeepPavlov-liter2 }
\end{figure}

\begin{figure}[H]
	\centering
	\includegraphics[width=0.9\linewidth]{image/res_DeepPavlov-liter_3 }
	\caption*{Рис. Г9. Результаты тестирования для модели DeepPavlov-liter\_3 }
	\label{fig:resDeepPavlov-liter3 }
\end{figure}

\begin{figure}[H]
	\centering
	\includegraphics[width=0.9\linewidth]{image/res_DeepPavlov-polit }
	\caption*{Рис. Г10. Результаты тестирования для модели DeepPavlov-polit }
	\label{fig:resDeepPavlov-polit }
\end{figure}

\begin{figure}[H]
	\centering
	\includegraphics[width=0.9\linewidth]{image/res_DeepPavlov-polit_2 }
	\caption*{Рис. Г11. Результаты тестирования для модели DeepPavlov-polit\_2 }
	\label{fig:resDeepPavlov-polit2 }
\end{figure}

\begin{figure}[H]
	\centering
	\includegraphics[width=0.9\linewidth]{image/res_DeepPavlov-polit_3 }
	\caption*{Рис. Г12. Результаты тестирования для модели DeepPavlov-polit\_3 }
	\label{fig:resDeepPavlov-polit3 }
\end{figure}

\begin{figure}[H]
	\centering
	\includegraphics[width=0.9\linewidth]{image/res_elmo-DeepPavlov-liter }
	\caption*{Рис. Г13. Результаты тестирования для модели elmo-DeepPavlov-liter }
	\label{fig:reselmo-DeepPavlov-liter }
\end{figure}

\begin{figure}[H]
	\centering
	\includegraphics[width=0.9\linewidth]{image/res_elmo-DeepPavlov-polit }
	\caption*{Рис. Г14. Результаты тестирования для модели elmo-DeepPavlov-polit }
	\label{fig:reselmo-DeepPavlov-polit }
\end{figure}


\end{document}