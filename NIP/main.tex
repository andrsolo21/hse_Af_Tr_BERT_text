%!TEX TS-program = xelatex

% Шаблон документа LaTeX создан в 2018 году
% Алексеем Подчезерцевым
% В качестве исходных использованы шаблоны
% 	Данилом Фёдоровых (danil@fedorovykh.ru) 
%		https://www.writelatex.com/coursera/latex/5.2.2
%	LaTeX-шаблон для русской кандидатской диссертации и её автореферата.
%		https://github.com/AndreyAkinshin/Russian-Phd-LaTeX-Dissertation-Template

\documentclass[a4paper,14pt]{article}

\input{data/preambular.tex}
\begin{document} % конец преамбулы, начало документа
	%\begin{titlepage}
	\begin{center}
		ФЕДЕРАЛЬНОЕ  ГОСУДАРСТВЕННОЕ АВТОНОМНОЕ \\
		ОБРАЗОВАТЕЛЬНОЕ УЧРЕЖДЕНИЕ ВЫСШЕГО ОБРАЗОВАНИЯ\\
		«НАЦИОНАЛЬНЫЙ ИССЛЕДОВАТЕЛЬСКИЙ УНИВЕРСИТЕТ\\
		«ВЫСШАЯ ШКОЛА ЭКОНОМИКИ»
	\end{center}
	
	\begin{center}
		\textbf{Московский институт электроники и математики}
		
		\textbf{Им. А.Н.Тихонова НИУ ВШЭ}
	\end{center}
	\vspace{1ex}	
	\begin{center}
		Подчезерцев Алексей Евгеньевич, группа БИВ172
		
		Солодянкин Андрей Александрович, группа БИВ172
	\end{center}	
	\vspace{1ex}
	\begin{center}
		\textbf{ПРОЕКТИРОВАНИЕ И РАЗРАБОТКА ЭЛЕКТРОННОЙ СИСТЕМЫ ДЛЯ ДОПОЛНИТЕЛЬНОГО ОБРАЗОВАНИЯ СО ШКОЛЬНИКАМИ}
	\end{center}	
	\vspace{2ex}
	\begin{center}
		Междисциплинарная курсовая работа \\
		по направлению 09.03.01 Информатика и вычислительная техника \\
		студентов образовательной программы бакалавриата \\
		<<Информатика и вычислительная техника>> \\

	\end{center}
	\vspace{2ex}
	\begin{flushright}
		Студент $\rule{5cm}{0.15mm}$ А.Е. Подчезерцев 
		
		\vspace{1ex}
		Студент $\rule{5cm}{0.15mm}$ А.А. Солодянкин 
	\end{flushright}
	\vspace{3ex}
	\begin{flushright}
		Руководитель\\
		Ассистент \\
		$\rule{5cm}{0.15mm}$ А.Ю. Ролич
	\end{flushright}
	\vfill
	\begin{center}
		Москва \the\year \, г.
	\end{center}
\end{titlepage}
	%\tableofcontents
	%\pagebreak
	
	\section{}
	Солодянкин Андрей;
	
	БИВ174;
	
	Исследование афинных преобразований в семантическом пространстве BERT.
	%\pagebreak
	
	\section{Актуальность}
	%2-3 страницы.
	
	%В последние годы компьютеры стали незаменимой частью нашей жизни, они представлены голосовыми помощниками, рекомендательными системами, системами умных домов, умными машинами и другими устройствами.
	%Важной частью всех этих систем являются модули, которые позволяют донести до компьютера 
	
	В последние годы технологии машинного обучения стали неотъемлемой частью нашей жизни.
	Они представлены голосовыми помощниками, рекомендательными системами, умными домами, умными автомобилями и другими системами.
	Важной частью этих систем являются модули, которые помогают сделать понятным для компьютера то, что от него требуется.
	Для систем по обработке текста это модули обработки естественного языка или Natural Language Processing (NLP).
	
	Направление обработки естественного языка активно развивается, последний большой прорыв был сделан в 2018 году командой Google AI. 
	Была представлена новая модель обработки естественного языка под названием BERT или Bidirectional Encoder Representations from Transformers \cite{bert}. 
	BERT продемонстрировал лучшее качество на тесте SQuAD (Stanford Question Answering Dataset) \cite{SQuAD} версии 1.1 для вопросно-ответных систем. На рисунке \ref{fig:quality_bert} представлены первые строчки таблицы лидеров для теста SQuAD 1.1 на момент выхода модели BERT.
	
	\begin{figure}[H]
		\centering
		\includegraphics[width=0.6\linewidth]{images/2018-11-03.-bert-nlp-method_2}
		\caption{Сравнение BERT с другими моделями на момент выхода модели}
		\label{fig:quality_bert}
	\end{figure}
	
	Сейчас появились модели с качеством лучше, чем у модели BERT. 
	При этом многие модели с хорошим качеством на второй версии теста SQuAD \cite{SQuAD} (SQuAD 2.0 \cite{SQuAD2}) используют те же принципы, что и модель BERT.
	На рисунке \ref{fig:quality140321} представлен топ 3 моделей на тесте SQuAD 2.0 \cite{SQuAD2} от 14.03.21.
	
	\begin{figure}[H]
		\centering
		\includegraphics[width=0.6\linewidth]{images/quality_14_03_21}
		\caption{Таблица лидеров теста SQuAD2.0 на 14.03.21}
		\label{fig:quality140321}
	\end{figure}

	При всей популярности модели BERT остаются аспекты которые, плохо изучены или вообще не изучены, например, влияние вложений друг на друга в одном предложении.
	Большинство людей работают с моделью, не понимая, что происходит в модели и как модель генерирует результаты.
	Происходит это из-за того, что трудно интерпретировать промежуточные данные.
	Кроме того, в промышленной сфере есть потребность в интерпретации моделей машинного обучения.
	
	Появляется потребность в интерпретации промежуточных данных.
	В нашей работе проверяется гипотеза о возможности аффинных преобразований для эмбедднгов слов в семантическом пространстве BERT и проводится оценка качества проведенных аффинных преобразований.
	
	
	

	\pagebreak
	
	\section{Обзор литературы}
	%минимум 30 работ
	%10-15 страниц

	\pagebreak
	
	\section{Постановка задачи}
	%0.5-1 страница
	%задачи (минимум 8)

	\pagebreak
	

	
	
	
	\newpage 
	\renewcommand{\refname}{{\normalsize Список использованных источников}} 
	\centering 
	
	%\bibliographystyle{gost780s} % ГОСТ 7.80
	%\bibliography{biblio} % MachLearn.bib
	
	\begin{thebibliography}{9} 
		\addcontentsline{toc}{section}{\refname} 
		\bibitem{bert}Devlin J. et al. Bert: Pre-training of deep bidirectional transformers for language understanding //arXiv preprint
		\bibitem{SQuAD}Rajpurkar P. et al. Squad: 100,000+ questions for machine comprehension of text //arXiv preprint arXiv:1606.05250. – 2016.
		\bibitem{SQuAD2}Rajpurkar P., Jia R., Liang P. Know what you don't know: Unanswerable questions for SQuAD //arXiv preprint arXiv:1806.03822. – 2018.
		
		
	\end{thebibliography}
	
\end{document} % конец документа
